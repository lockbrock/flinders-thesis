%%%%%%%%%%%%%%%%%%%%%%%%%%%%%%%%%%%%%%%%%
% Masters/Doctoral Thesis 
% LaTeX Template
% Version 2.5 (27/8/17)
%
% This template was downloaded from:
% http://www.LaTeXTemplates.com
%
% Version 2.x major modifications by:
% Vel (vel@latextemplates.com)
%
% This template is based on a template by:
% Steve Gunn (http://users.ecs.soton.ac.uk/srg/softwaretools/document/templates/)
% Sunil Patel (http://www.sunilpatel.co.uk/thesis-template/)
%
% Template license:
% CC BY-NC-SA 3.0 (http://creativecommons.org/licenses/by-nc-sa/3.0/)
%
%%%%%%%%%%%%%%%%%%%%%%%%%%%%%%%%%%%%%%%%%

%----------------------------------------------------------------------------------------
%	PACKAGES AND OTHER DOCUMENT CONFIGURATIONS
%----------------------------------------------------------------------------------------

\documentclass[
12pt, % The default document font size, options: 10pt, 11pt, 12pt
oneside, % Two side (alternating margins) for binding by default, uncomment to switch to one side
english, % ngerman for German
onehalfspacing, % Single line spacing, alternatives: onehalfspacing or doublespacing
%draft, % Uncomment to enable draft mode (no pictures, no links, overfull hboxes indicated)
%nolistspacing, % If the document is onehalfspacing or doublespacing, uncomment this to set spacing in lists to single
%liststotoc, % Uncomment to add the list of figures/tables/etc to the table of contents
%toctotoc, % Uncomment to add the main table of contents to the table of contents
parskip, % Uncomment to add space between paragraphs
%nohyperref, % Uncomment to not load the hyperref package
headsepline, % Uncomment to get a line under the header
%chapterinoneline, % Uncomment to place the chapter title next to the number on one line
%consistentlayout, % Uncomment to change the layout of the declaration, abstract and acknowledgements pages to match the default layout
]{MastersDoctoralThesis} % The class file specifying the document structure

\usepackage[utf8]{inputenc} % Required for inputting international characters
\usepackage[T1]{fontenc} % Output font encoding for international characters
\usepackage{todonotes}
\usepackage{mathpazo} % Use the Palatino font by default
\usepackage{listings}
\usepackage{color}
\usepackage{tabularx}
\usepackage{float}

\usepackage[style=authoryear, dashed=false]{biblatex} % Use the bibtex backend with the authoryear citation style (which resembles APA)
%\usepackage{biblatex}
\bibliography{main} 
\renewcommand*{\nameyeardelim}{\addcomma\space}

\usepackage[autostyle=true]{csquotes} % Required to generate language-dependent quotes in the bibliography


%----------------------------------------------------------------------------------------
%	MARGIN SETTINGS
%----------------------------------------------------------------------------------------

\geometry{
	paper=a4paper, % Change to letterpaper for US letter
	inner=2.5cm, % Inner margin
	outer=3.8cm, % Outer margin
	bindingoffset=.5cm, % Binding offset
	top=1.5cm, % Top margin
	bottom=1.5cm, % Bottom margin
	%showframe, % Uncomment to show how the type block is set on the page
}

%----------------------------------------------------------------------------------------
%	THESIS INFORMATION
%----------------------------------------------------------------------------------------
\thesistitle{Developing diagnostic tools for the Serval Mesh Network} % Your thesis title, this is used in the title and abstract, print it elsewhere with \ttitle
\supervisor{Dr. Paul \textsc{Gardner-Stephen} \linebreak Dr. Saeed \linebreak \textsc{Rehman}} % Your supervisor's name, this is used in the title page, print it elsewhere with \supname
\examiner{} % Your examiner's name, this is not currently used anywhere in the template, print it elsewhere with \examname
\degree{Bachelor of Engineering(Software)(Honours)} % Your degree name, this is used in the title page and abstract, print it elsewhere with \degreename
\author{Lachlan Brock} % Your name, this is used in the title page and abstract, print it elsewhere with \authorname
\addresses{} % Your address, this is not currently used anywhere in the template, print it elsewhere with \addressname

\subject{Software Engineering} % Your subject area, this is not currently used anywhere in the template, print it elsewhere with \subjectname
\keywords{} % Keywords for your thesis, this is not currently used anywhere in the template, print it elsewhere with \keywordnames
\university{\href{https://www.flinders.edu.au/}{Flinders University}} % Your university's name and URL, this is used in the title page and abstract, print it elsewhere with \univname
\department{\href{https://www.flinders.edu.au/college-science-engineering}{College of Science and Engineering}} % Your department's name and URL, this is used in the title page and abstract, print it elsewhere with \deptname
\group{} % Your research group's name and URL, this is used in the title page, print it elsewhere with \groupname
\faculty{\href{}{}} % Your faculty's name and URL, this is used in the title page and abstract, print it elsewhere with \facname

\defFirstResearchQuestion{What is the state of LBARD and the test framework?}
\defSecondResearchQuestion{What other network testing tools exist and what features do they have?}
\defThirdResearchQuestion{What additional tools and features would be useful for analysing and testing the functionality of the Serval Mesh?}
\defFourthResearchQuestion{How might these additional diagnostic tools be created and evaluated?}

\AtBeginDocument{
\hypersetup{pdftitle=\ttitle} % Set the PDF's title to your title
\hypersetup{pdfauthor=\authorname} % Set the PDF's author to your name
\hypersetup{pdfkeywords=\keywordnames} % Set the PDF's keywords to your keywords
}

\begin{document}

\frontmatter % Use roman page numbering style (i, ii, iii, iv...) for the pre-content pages

\pagestyle{plain} % Default to the plain heading style until the thesis style is called for the body content

%----------------------------------------------------------------------------------------
%	TITLE PAGE
%----------------------------------------------------------------------------------------

\begin{titlepage}
\begin{center}

\vspace*{.06\textheight}
{\scshape\LARGE \univname\par}\vspace{1.5cm} % University name
\textsc{\Large Honours Thesis}\\[0.5cm] % Thesis type

\HRule \\[0.4cm] % Horizontal line
{\huge \bfseries \ttitle\par}\vspace{0.4cm} % Thesis title
\HRule \\[1.5cm] % Horizontal line
 
\begin{minipage}[t]{0.4\textwidth}
\begin{flushleft} \large
\emph{Author:}\\
{\authorname} % Author name - remove the \href bracket to remove the link
\end{flushleft}
\end{minipage}
\begin{minipage}[t]{0.4\textwidth}
\begin{flushright} \large
\emph{Supervisors:} \\
{\supname} % Supervisor name - remove the \href bracket to remove the link  
\end{flushright}
\end{minipage}\\[3cm]
 
\vfill

\large \textit{A thesis submitted in fulfilment of the requirements\\ for the degree of \degreename}\\[0.3cm] % University requirement text
%%\groupname\\\deptname\\[2cm] % Research group name and department name
 
\vfill

{\large \today}\\[4cm] % Date
%\includegraphics{Logo} % University/department logo - uncomment to place it
 
\vfill
\end{center}
\end{titlepage}

%----------------------------------------------------------------------------------------
%	DECLARATION PAGE
%----------------------------------------------------------------------------------------

\begin{declaration}
\addchaptertocentry{\authorshipname} % Add the declaration to the table of contents
\noindent I, \authorname, declare that this thesis titled, \enquote{\ttitle} and the work presented in it are my own. I confirm that:

\begin{itemize} 
\item This work was done wholly while in candidature for a degree of \degreename.
\item This document is in accordance with the plagiarism policy of \univname.
\item Where any part of this thesis has previously been submitted for a degree or any other qualification at this University or any other institution, this has been clearly stated.
\item Where I have consulted the published work of others, this is always clearly attributed.
\item Where I have quoted from the work of others, the source is always given. With the exception of such quotations, this thesis is entirely my own work.
\item I have acknowledged all main sources of help.
\item Where the thesis is based on work done by myself jointly with others, I have made clear exactly what was done by others and what I have contributed myself.\\
\end{itemize}
 
\noindent Signed:\\
\rule[0.5em]{25em}{0.5pt} % This prints a line for the signature
 
\noindent Date:\\
\rule[0.5em]{25em}{0.5pt} % This prints a line to write the date
\end{declaration}

\cleardoublepage

%----------------------------------------------------------------------------------------
%	QUOTATION PAGE
%----------------------------------------------------------------------------------------

% \vspace*{0.2\textheight}

% \noindent\enquote{\itshape Add quotation later}\bigbreak

% \hfill Add who I attribute it to

%----------------------------------------------------------------------------------------
%	ABSTRACT PAGE
%----------------------------------------------------------------------------------------

\begin{abstract}
  \addchaptertocentry{\abstractname} % Add the abstract to the table of contents
The Serval Mesh network is an infrastructure-less network designed for use in disaster-recovery efforts.
To ensure its reliable functioning during deployment, the network must be thoroughly tested.
To test the network, the Serval team have developed a software emulator to allow them to test the network in various scenarios.

This emulator has several limitations however, and is unable to provide clear and chronologically ordered log files or any graphical output to testers.
This thesis covers the implementation of configurable network devices, expanding the capabilities of the frameworks to model the Serval Mesh.
Additionally, tools are developed to generate simpler and easily understood log files, and produce graphical representations of the network traffic throughout a test.

With these tools, several bugs were uncovered in core Serval code that limit its effectiveness in field deployments.
Further, the basis for comparing the emulated Serval Network and real-world tests was outlined for future developers to analyse the accuracy of this emulation.
With the development of these tools, the aims are furthered of the Serval Project to produce a reliable network to help save lives in disaster recovery efforts.


\end{abstract}

%----------------------------------------------------------------------------------------
%	ACKNOWLEDGEMENTS
%----------------------------------------------------------------------------------------

\begin{acknowledgements}
\addchaptertocentry{\acknowledgementname} % Add the acknowledgements to the table of contents

First, I would like to thank my both of my supervisors, Paul and Saeed for their support and assistance throughout this year.
Of particular note, I would like to thank Paul for his long-distance guidance and career advice from Arkaroola, and thank you for dealing with poor microphone quality and satellite-bandwidth Skype calls.


I would also like to thank Josh, Phil, and Maddi.
You have all provided some much-needed sanity throughout this year, particularly during the mad rush of end-year thesis writing.


A major thank you to my parents, this has been a crazy and stressful year — even without the Honours year, and I have nothing but thanks for the support and understanding you have had for me.

Finally, I would like to thank my girlfriend, Sierra. 
You have been an absolute pillar of support throughout this year, and I honestly don't know if I could have done this without you by my side.

\end{acknowledgements}

%----------------------------------------------------------------------------------------
%	LIST OF CONTENTS/FIGURES/TABLES PAGES
%----------------------------------------------------------------------------------------

\tableofcontents % Prints the main table of contents

\listoffigures % Prints the list of figures

%\listoftables % Prints the list of tables

%----------------------------------------------------------------------------------------
%	ABBREVIATIONS
%----------------------------------------------------------------------------------------

%\begin{abbreviations}{ll} % Include a list of abbreviations (a table of two columns)

%\textbf{LAH} & \textbf{L}ist \textbf{A}bbreviations \textbf{H}ere\\
%\textbf{WSF} & \textbf{W}hat (it) \textbf{S}tands \textbf{F}or\\

%\end{abbreviations}

%----------------------------------------------------------------------------------------
%	PHYSICAL CONSTANTS/OTHER DEFINITIONS
%----------------------------------------------------------------------------------------

%\begin{constants}{lr@{${}={}$}l} % The list of physical constants is a three column table

% The \SI{}{} command is provided by the siunitx package, see its documentation for instructions on how to use it

%Speed of Light & $c_{0}$ & \SI{2.99792458e8}{\meter\per\second} (exact)\\
%Constant Name & $Symbol$ & $Constant Value$ with units\\

%\end{constants}

%----------------------------------------------------------------------------------------
%	SYMBOLS
%----------------------------------------------------------------------------------------

%\begin{symbols}{lll} % Include a list of Symbols (a three column table)

%$a$ & distance & \si{\meter} \\
%$P$ & power & \si{\watt} (\si{\joule\per\second}) \\
%Symbol & Name & Unit \\

%\addlinespace % Gap to separate the Roman symbols from the Greek

%$\omega$ & angular frequency & \si{\radian} \\

%\end{symbols}

%----------------------------------------------------------------------------------------
%	DEDICATION
%----------------------------------------------------------------------------------------

%\dedicatory{For/Dedicated to/To my\ldots} 

%----------------------------------------------------------------------------------------
%	THESIS CONTENT - CHAPTERS
%----------------------------------------------------------------------------------------

\mainmatter % Begin numeric (1,2,3...) page numbering

\pagestyle{thesis} % Return the page headers back to the "thesis" style

% Include the chapters of the thesis as separate files from the Chapters folder
% Uncomment the lines as you write the chapters

% Chapter Template

\chapter{Introduction}\label{chapter:firstchapter} % Main chapter title

\label{Chapter1} % Change X to a consecutive number; for referencing this chapter elsewhere, use \ref{ChapterX}

The introduction should provide the following:
\begin{itemize}
    \item background to the topic
    \item brief review of current knowledge (this is not the literature review –it’s only a high level overview)
    \item state hypotheses
    \item indicate gaps in knowledge, state aims of the thesis and how it fits into the gap
    \item an outline of the followingchapters.
\end{itemize}


The introduction should follow the recommended structure:
\begin{itemize}
    \item state the general background of the thesis topic and give some background
    \item provide an overview of the literature related to the thesis topic
    \item define the terms and scope of thethesis
    \item outline the current situation
    \item evaluate the advantages/disadvantages of existing solutions and identify the knowledge gap
    \item identify the importance of the proposed research
    \item state the research problem/questions
    \item state the research aims and/or research objectives
    \item state the hypotheses
    \item outline the experiment methodology
    \item outline the structure of thethesis
\end{itemize}


\emph{Any time a decision is made, write it down}
\begin{itemize}
    \item Take notes
    \item Git commits of meaningful info
    \item If in doubt, write too much
    \item Answer all the why questions
\end{itemize}

%----------------------------------------------------------------------------------------
%	SECTION 1
%----------------------------------------------------------------------------------------

\section{The Serval Project}\label{sec:firstsection}
Add history, what it is, etc.



\subsection{Aims}



\subsection{How it works}
An overview of how the serval project works


%----------------------------------------------------------------------------------------
%	SECTION 2
%----------------------------------------------------------------------------------------

\section{Project Overview}

\begin{itemize}
    \item What the project is (briefly)
    \item Why the project needs to be done (no current large-scale network testing)
    \item Allows visual analysis of tests 
    \item Current test framework is limited in scale, doesn't have many features relevant to real-world: (multi-radio technology not possible)
    \item 
\end{itemize}
% Chapter Template

\chapter{Literature Review} % Main chapter title

\label{Chapter2} % Change X to a consecutive number; for referencing this chapter elsewhere, use \ref{ChapterX}

%----------------------------------------------------------------------------------------
%	SECTION 1
%----------------------------------------------------------------------------------------

\section{Mesh Networks}
A Mesh Network is a type of wireless network where each device directly connects to any neighbouring devices within range.
To communicate over longer distances than a single device can communicate, each device in the network — called a node — act as a relay, where they forward any data they receive to their neighbouring nodes \parencite{akyildiz_wang_2009}.

Mesh networks function without any central hierarchy, allowing for networks to self-organise and automatically create a network with any compatible surrounding nodes \parencite{madhusudan_2019}.

Due to the non-hierarchical and self-configuring nature, wireless mesh networks have "low up-front costs, easy network maintenance, robustness, and reliable service coverage" \parencite{akyildiz_wang_2009}.
This allows for quick and easy deployment of a mesh network in unfamiliar locations, as the nodes will automatically connect with any neighbours, and form a topology without any manual configuration required.

\section{The Serval Mesh Network}

\subsection{The Serval Project}
In 2010, Dr. Paul Gardner-Stephen began the development of the Serval Mesh Network in response to the earthquakes in Haiti \parencite{gardner2011serval}.
Originally, the network was designed to connect smartphone devices together using WiFi and Bluetooth technology to form an ad hoc mesh network.
However, through development this was abandoned due to the lack of ad hoc WiFi support in Android and iOS devices \parencite{productizingServalMesh}.
To circumvent this, the first generation Serval Mesh Extender was developed.
This Mesh Extender was built using a GL-AR150 portable WiFi router, and was able to reach ranges of up to 100 meters.
With this Mesh Extender, users could connect their smartphone devices to the wireless hotspot of the Mesh Extender without requiring their smartphones themselves to become a node in the mesh network.
To extend the range of these Mesh Extenders further than the 802.11 WiFi would allow, a radio module was added to the Mesh Extender \parencite{productizingServalMesh}.
The RFD900 radio module allowed for Serval Mesh Extenders to reach ranges of up to 3 kilometres over the 915MHz radio band.


After several iterations, the latest version of the Serval Mesh Extender features both inbuilt WiFi and UHF radio over the RFD900+ packet radio.
In 2017 these Mesh Extenders were deployed in eight different locations around Vanuatu in the first large-scale test of the Serval Mesh.
\parencite{pilotingServalMesh}


\subsection{Rhizome}
The Rhizome protocol was developed for the Serval Mesh Network, and forms the basis of all services provided by the Serval Mesh \parencite{rhizomeDocumentation}.

Rhizome has several features that are useful for the Serval Mesh network.
The first of these is that it functions as a store and forward protocol, where when data is transferred across the mesh network, this data is stored on the Serval node that this data is transferred to, and is then forwarded on to any neighbouring nodes that do not have this piece of data \parencite{gardner2011serval}.
In practice, this means that the Serval Mesh is very rugged, as the data sent through the Mesh will be stored in multiple locations so that even if one node is removed from the network due to a situation such as a natural disaster or power outage, then this data will still — eventually — reach its destination as long as a path can be made to the destination.


Further, the decentralised nature of Rhizome means that there is no central point of failure or central location where the networks data is stored; it is stored on every node \parencite{servalWiFiMultiModel}.
This is vitally important in situations such as disaster recovery where the Serval Mesh will be used in areas where it is possible that a Rhizome node may be lost.

Rhizome uses two data structures, a manifest and a payload.
A Rhizome payload is simply a unit of data such as an image file, or a message.
A manifest lists all meta-data associated with this payload, including the size, data type, date of creation, and intended destination \parencite{gardner2011serval}.
A Rhizome Bundle is formed of a manifest and a payload. 

\subsection{Serval-DNA}
Serval-DNA forms the main software component of the Serval Mesh Network.
Written in C, Serval-DNA handles all the core functionality of the Serval Network and provides the implementation of the Rhizome protocol \parencite{servalMesh2013}. 
Serval-DNA handles the transfer of Rhizome Bundles over WiFi automatically, determining what bundles are required by nearby nodes through a synchronisation method.

\subsection{LBARD}
The Low Bandwidth Asynchronous Rhizome Demonstrator (LBARD) is a C program written for the Serval Project that handles synchronising Rhizome Bundles over low bandwidth connections such as those used in the RFD900 packet radio in the Serval Mesh Extender.
LBARD was developed to run as an addition to Serval-DNA and provides the functionality to communicate with neighbouring Serval nodes as well as implementing the radio drivers necessary to use the various radio types that LBARD supports.

Since the default synchronisation method for Rhizome as implemented in Serval-DNA relies on a relatively high amount of bandwidth in comparison with the low bandwidth available on UHF or HF radios a new synchronisation method needed to be used for these low bandwidth links \parencite{productizingServalMesh}.
LBARD uses a tree-synchronisation protocol that uses a XOR of Rhizome bundle hashes to quickly determine which bundles are missing from its neighbours \parencite{lbardDocumentation}.
This synchronisation protocol allows for LBARD nodes to efficiently determine if a neighbour is missing any of the bundles that it has in its Rhizome store with a minimal amount of communication needed.
This is crucial for the low-bandwidth links that exist between two LBARD nodes, as it allows for LBARD to utilise the rest of the bandwidth that is available to it to transfer the Rhizome bundles.
After the LBARD node has determined what bundles are required to be transferred, these bundles are then transferred in 64-byte pieces \parencite{lbardDocumentation}.
To ensure that these pieces are received, LBARD continues sending these pieces until it receives a confirmation from the receiving node that all the pieces have been received intact.

As it stands, LBARD is able to support several radio types, including the RFD900, with some experimental implementations for the CodanHF and BarrettHF radios, as well as a prototype Outernet satellite uplink \parencite{lbardDocumentation}.

\subsection{Serval Test Framework}
The Serval Test Framework is a test suite designed for the Serval Network, that provides the framework for emulated Serval nodes to be tested.
The test framework is largely written as Bash scripts that setup specified tests, and links Serval nodes together using simulated WiFi or radio connections \parencite{servalTestDocumentation}.
The test framework is divided into two parts; the Serval-DNA framework, and the LBARD framework.
The framework provides the tools to test a wide range of Serval functionality, as it is able to directly monitor the performance and responses of the software and compare it to the expected output as defined in the test definition.
The framework functions as an emulator rather than a simulator as it is running the real Serval-DNA/LBARD software, with only the links between nodes simulated \parencite{servalTestDocumentation}.
The Serval Test Framework will be covered in more detail in Chapter 3.

\subsection{Serval Test Network}
The Serval Test Network is a laboratory network that has been set up at the Flinders University Campus.
This network features 14 Mesh Extenders with 10 of these attached to Serval Mesh Observers \parencite{wade_2019}.
These observers log all WiFi and UHF data that is sent and received by these extenders, which is then logged and sent back to a single main server alongside the logs of these extenders \parencite{lancaster_2019}.
This allows for real-world tests to be run on this test network, with the test data sent back to a single computer for analysis.


\section{Network Testing}
When testing networks, two main routes exist; software emulation, and real-world test beds.

As opposed to network simulation, which serves as a mathematical model of how a network may function under certain criteria, software emulation runs the real software that is to be used on each of the nodes of the network and simulates the network links between these nodes \parencite{comparingSimulationTestbeds}.
Software emulation is particularly useful in testing mesh networks, as these can networks are often more difficult to develop real-world test beds for due to their non-hierarchical nature \parencite{predeploymentTesting2006}.
Further, emulation allows for testers to easily change the network layout with minimal extra effort or cost, while a real-world hardware test-bed would require far more effort to change this layout, and adding a node may come at an extra cost.
Network emulation requires a high-level of simulation fidelity to accurately model and test a network.
Simulation fidelity is the accuracy of a network to model the corresponding real-world network, and requires the relative accuracy of information such as percentage of packet loss and signal strength in the simulation.

Real-world test beds consist of a real deployed network that has some tests run on it \parencite{testingWirelessMesh2010}.
These tests are often smaller in scale than emulated tests due to the relative difficulty and additional cost of setting up real networks.
These tests are often more accurate than an emulated test network, as these test beds form the same or similar network to a real-world scenario \parencite{disasterResilientMesh2013}.


\section{Features of Popular Network Emulators/\\Simulators}
\subsection{C.O.R.E}
Common Open Research Emulator (CORE) is an open source emulator produced by the U.S. Naval Research Laboratory, and is considered an industry-standard network emulator \parencite{coreDocumentation}.
CORE offers several key features for testing different networks. Some of these key features and a brief explanation are listed below:
\begin{itemize}
    \item \textbf{Network visualisation} showing the traffic throughout the network. This can be played back after a test concludes.
    \item \textbf{Test Report} detailing the activity on each of the network nodes throughout the test.
    \item \textbf{Node specific configuration} allowing for fine control over the specific configuration of a certain node.
\end{itemize}

\subsection{The Network Simulator (ns-3)}
The Network Simulator (ns-3) is a network simulator often used in research to model networks and test specific networking protocols \parencite{modellingAndTools2010}.
The ns-3 simulator features several features for testers to simulate a variety of networks. Some of these features are listed below:
\begin{itemize}
    \item \textbf{Animated network visualisation} to display the network traffic throughout the test.
    \item \textbf{Support for various link types} including the ability to define variables such as packet loss and signal strength.
\end{itemize}


\section{Summary}
Testing networks before real-world deployment is crucial for determining their reliability and locating flaws \parencite{predeploymentTesting2006}.
This is of particular importance if these networks are mission-critical, as is the case of the Serval Mesh Network.
This chapter has answered the second research question "\secondRQ" and has begun to answer the first research question "\firstRQ".
In the next chapter, this first research question will continue to be answered through an analysis of the Serval-DNA and LBARD test frameworks.
 
\chapter{Description and analysis of existing test framework} 
% Main chapter title

\label{Chapter3} % Change X to a consecutive number; for referencing this chapter elsewhere, use \ref{ChapterX}

%----------------------------------------------------------------------------------------
%	SECTION 1
%----------------------------------------------------------------------------------------

\section{Description of existing test framework}
Description and analysis of existing LBARD framework

\begin{itemize}
    \item Go into much more detail than lit review
    \item Since we are extending, need to fully explain and understand
    \item Analysis of log files
    \item how it functions
\end{itemize}

Two emulators for the Serval mesh exist. 
The first was built to test the functionality of the Serval mesh in its early days; when nodes were purely communicating over WiFi. 
This emulator is able to emulate multiple aspects of the serval-dna software, from validating the database integrity to \todo{add more} to simulating WiFi communication between Serval nodes. 
Later, once LBARD was added to the Serval project, a second emulator was built. 
This emulator performed similar functions to that of the serval-dna; testing internal LBARD functionality, and emulating radio communication between Serval nodes. 
While the LBARD emulator serves as an extension to the serval-dna emulator - extending and overwriting functionality - there is no ability to run tests that feature both WiFi and radio links.
\\
\emph{Unless otherwise specified, when talking about the test framework in this document, we will be talking about the LBARD test framework.}
\todo{add: what it is, how it functions}
\\


The two emulators are both written as bash shell scripts, where the LBARD test framework extends and overwrites several of the functions in the serval-dna emulator to get it to work with LBARD and fakeradio.
\todo{Add more about how it extends it, what the serval-dna test script does, etc.}
LBARD Emulator
\begin{itemize}
    \item Bash script
    \item 
\end{itemize}



\subsection{Aims of test framework}
What it /wants/ to be doing

Why it exists


\subsection{Functionality}
The LBARD test framework is able to run a vast range of defined tests, and because of this, is able to test large portions of the functionality of the LBARD software.

The test suite runs multiple instances of the actual serval-dna software, with LBARD processes running on top of this, and each instance is networked together through the 'fakeradio' software.
The fakeradio software monitors the LBARD radio outputs of each of the LBARD interfaces, and sends the packets it receives to the inputs of each of the LBARD radio input files, dependent on the rules defined when starting the fakeradio program.
The rules that can be sent to fakeradio allow test creators to define network layouts by denying all traffic between nodes unless specified otherwise, allowing for any topology to be defined.


When a test is run, the test suite starts the specified number of serval instances, each with their own separate databases, configurations, and log files.
Then, LBARD instances are started for each of these instances, running simulated radio hardware (RFD900, BarrettHF, etc.) as defined in the test definition.
Finally, fakeradio is started and begins to listen to the output files of each of the LBARD interfaces.
Once a test is concluded, the framework collects all of the log files and stdin/stdout/stderr of the serval intstances, LBARD instances, and fakeradio.
These are then collated into a single test log file, with every piece of data timestamped.
This way of running the tests has several benefits for the purpose of this Honours project; firstly, the test framework is running the real software that runs in a real-world scenario, with only the radio and wifi communications being simulated; secondly, the framework collates all of this data into a single log file, greatly minimising the work that needs to be done in later stages of this project to extract events in a test from multiple log files.

\subsection{Defining Tests}
Tests are defined in either the lbard or lbard\_size\_tests file. 
These are both bash files that source the lbard test framework, and simply define what tests are to be run.
Tests are comprised of three basic components; a doc string with a brief description of what the test does, a 'setup' function that sets up the test environment, and finally, a 'test' function, that is run when the test is run.
The doc string is used when running tests and serves as feedback to the tester as to what test is currently being run.
The setup function is run before the test is conducted. This function serves to set up any necessary configurations for the running of this test. This is where the fakeradio rules are defined, files added to Serval instances, and any other necessary setup for a test.
Finally in the test function, the conditions for a successful test are defined. Once this function completes (or an error/fail/timeout is encountered), the test concludes.


This can be seen in Figure \ref{fig:testDefinition}. 
In this example, setup function defines the fakeradio rules, then adds a single file of 50 bytes to the instance A. 
In the test function, a function 'all\_bundles\_received' is defined, that simply checks if instance B received a bundle with the specified bundle ID and version. 
The test then waits until this bundle is received. 
If this bundle is received before the default timeout, the test will pass.

\begin{figure}
    \begin{centering}
        \includegraphics[width=14cm,height=20cm,keepaspectratio]{Figures/testdefinitionexample.png}
        \caption{Example of LBARD test definition}
        \label{fig:testDefinition}
    \end{centering}
\end{figure}


\begin{itemize}
    \item \# Intro, brief explanation
    \item \# The features it has
    \begin{itemize}
        \item \# Logging all info
        \item \# run the actual software
    \end{itemize}
    \item \# Defining tests
    \item What it currently does (what tests can do)
    \item What tests already exist (show all? list notable?)
    \item How the tests work
    \begin{itemize}
        \item Setting up virtual devices
        \item Running serval-dna test suite underneath
        \item Simulates links and traffic between
        \item Logs data transfer along those links
    \end{itemize}
    \item Running the tests (feedback, PASS/FAIL/ERROR)
\end{itemize}


\subsection{Outputs}
When a test is run a log file is produced. 
This log file contains the outputs from multiple processes. 
The log file contains the outputs (stdout) and error messages (stderr) from: \begin{itemize}
    \item the test script,
    \item fakeradio and, 
    \item all the fakeradio and LBARD forks, 
    \todo{Can we also get logs from the WiFi transfer?}
\end{itemize}

As we can see, all components of the LBARD test framework are logged.
The end result of this is that a single log file with the information from all components of the framework.
\todo{Rewrite this garbage}



\begin{itemize}
    \item Log files
    \item Format of them
    \item What information is found within
\end{itemize}



\section{Limitations}
What it doesn't do
\begin{itemize}
    \item No ability to have dual link types; ie. a node cant send/receive WIFI as well as send/receive UHF. Means that all topologies are limited to a single type of radio type. \emph{Not realistic at all for a real-world situation}
    \item Cannot currently use the serval-dna WiFi simulator ALONGSIDE LBARD-tests fakeradio
    \item Does not output topology information
\end{itemize}

\section{How to improve}
\begin{itemize}
    \item Talk about what I'll be doing to improve the framework
    \item What features it needs to suit this honours
\end{itemize}


\begin{itemize}
    \item Add new lbard/tests file specifically for topologies: no point in rendering .dot files for the small setup stuff
    \item Add to topology test file the ability to have multiple link types: we're doing combination radio types in the real world so we need to cover that
\end{itemize}

% Chapter Template

\chapter{Extending the Test Framework} % Main chapter title
\label{Chapter4}

%----------------------------------------------------------------------------------------
%	SECTION 1
%----------------------------------------------------------------------------------------

\begin{itemize}
    \item Modify test framework
    \item Modify fakeradio to output layout
    \item Get dual link types working
    \item Be able to define different WiFi layouts
    \item 
\end{itemize}


\begin{itemize}
    \item Currently, no way to easily determine what topology is in machine readable format
    \item What info is needed for this 
    \item How can we represent it?
    \item Why format?
    \item JSON vs CSV
    
\end{itemize}


%-----------------------------------
%   SECTION 1
%-----------------------------------
\section{Joint wifi \& radio tests}
A key part of the test framework that was missing was support for networks with both wifi and radio link types. 
This presents a major limitation  to the test framework as real Serval networks would feature both of these network links.
To implement this feature, several additions need to be made to the test framework.

First, the ability to define which link simulation will be used for each nodes needs to be defined.
This allows test definition writers to define the functionality(??) of each of the nodes. 
For instance, test writers are able to define if a specific node in a topology is using the fakeradio tool or simply acting as wifi, or even, both. 

Next, the ability to define specific network topologies will need to be added.
This already exists in the case of the fakeradio networks as discussed in the previous chapters, this is not easily done with the simulated wifi tool.

Once these two pieces of functionality have been added to the test framework, developers will be able to test considerably larger and more complicated network topologies with ease.
With this expanded test framework the Serval team will be able to significantly improve their knowledge of the Serval network; complicated, larger, and reproducable serve a huge benefit to the goals of the Serval team. \emph{rewrite, this is bad}



What needed to be done for this?
\begin{itemize}
    \item Add capability to define interface to use
    \item Strip definitions to get lists of what interfaces for each node
    \item For each node, add the interfaces that are in lists
    \item For each node, add appropriate configuration
    \item Pass lbard rules to fakeradio as per normal
\end{itemize}


\subsection{Subsection 2}


%----------------------------------------------------------------------------------------
%	SECTION 2
%----------------------------------------------------------------------------------------

\section{Retreiving network layouts}
To be used in next section when we are creating diagrams from the network layouts
 
% Chapter Template

\chapter{Improving the Output} % Main chapter title

\label{Chapter5} % Change X to a consecutive number; for referencing this chapter elsewhere, use \ref{ChapterX}

%----------------------------------------------------------------------------------------
%	SECTION 1
%----------------------------------------------------------------------------------------
\begin{itemize}
    \item Create simple log
    \item Create tool to render graphics
    \item Example images
\end{itemize}

\begin{itemize}
    \item \emph{How to get render the output: what to do?}
    \item First, create a simple log
    \item Get the layout
    \item Create dot file of layout
    \item Sort by major and minor events
\end{itemize}

\textbf{Add intro. Why we're doing this, etc.}

\section{Creating a Simple Log}
\begin{itemize}
    \item \#Reasoning behind doing this
    \item \#Loop through file
    \item \#Filter each line, return appropriate line type
    \item \#Print in proper format
    \item   Add issues: LBARD T+ etc.
    \item \#Show development of output format
    \item \#Talk about what events, why they were selected
    \item \#Show output log screenshot
    \item Link with the test - run in the 'finally' section
\end{itemize}

The first step in improving the output of the test framework is improving the log output.
To do this, it was decided that a simple log file should be created for several reasons.
The log files produced by the test framework have a considerable amount of detail contained within them, however this comes at the cost of being incredibly large - even a relatively simple test such as DualType (produced in the previous section) can create log files of 4,000+ lines.
More than that, the log files produced are very difficult to follow, as they are ordered by log file, then chronologically.
Meaning that following a chain of events between multiple nodes becomes a process of consistently switching between sections of log file.
\todo{Add more; formatting, timestamp, etc.}

To fix this, the simple log will require six main features.
\begin{itemize}
    \item Consistent formatting
    \item Chronological ordering by timestamp
    \item Reduce amount of information to only useful info
    \item Machine \emph{and} human readable layout information
    \item Log traceability to original log file
    \item Support all pre-existing topology tests
\end{itemize}


Several options exist for creating a simple log.
The first is to modify the pre-existing codebase of LBARD, Servald, Fakeradio, and the test framework, and modify their logging to meet the above requirements.
This is not considered to be a useful use of this thesis, as this would require changing large parts of the codebase for these programs for a relatively mild improvement, as well as modifying the code of pre-tested and running devices.

The next is to modify the test framework to process and modify the log lines while the tests are run before piping them to the log file.
While this does solve the issue of the previous proposed solution this would add a large amount of bloat to the test framework that is not required in the vast majority of the tests, and may actually prove to be either impossible or vastly difficult using the Shell scripts of the test framework.
Further, this will increase the run-time for tests unnecessarily.

The final option is to write a program that after a test is run and the log file produced, processes the log file and filters and simplifies it, and outputs a separate, simpler log file. 
This is the method that was undertaken in this thesis. 
The program will be written in C as it needs to be fast, portable, and use minimal external dependencies, as well as it is the language that the majority of the Serval codebase is written in - allowing for future developers to easily maintain and improve the code.

As shown in \figurename{ \ref{fig:chapter5SimpleFlowchart}} the program follows a simple structure.
The specified log file is opened, and the first line is read into memory.
This line is then filtered depending on its content; if the line should be in the simple log, and what type of line it is. \todo{Fix this sentence}
The filtered line is then modified to be in a consistent format, and then the formatted line is written to the output simple log file.
The program repeats these steps with each line in the file until it reaches the end of the file.

\begin{figure}
    \begin{centering}
        \includegraphics[width=10cm,height=20cm,keepaspectratio]{Figures/Chapter5-SimpleLogFlowchart.png}
        \caption{Flowchart of creating the simple log}
        \label{fig:chapter5SimpleFlowchart}
    \end{centering}
\end{figure}

\subsection{Filtering Lines}
\begin{itemize}
    \item \#What lines we're selecting
    \item Why they are chosen
\end{itemize}

To ensure that the simple log contains only important lines, the input file needs to be filtered.
Filtering the log file allows for the simple log to contain only lines that are crucial to understanding the operating of the test without overwhelming the person examining the log file.

To determine the minimal information required to understand the tests, the output log files of the tests were analysed.
While analysing the output log files, the following list of important events was created.
\begin{itemize}
    \item \textbf{Setup} Start of a process, lbard/fakeradio
    \item \textbf{Setup} Test details
    \item \textbf{Setup} Layout information (WiFi and Fakeradio)
    \item \textbf{Setup} SID information
    \item \textbf{Servald} Sending and receiving packets
    \item \textbf{Servald} Adding manifest
    \item \textbf{LBARD} Neighbour has a bundle
    \item \textbf{LBARD} Send and receive bundles
    \item \textbf{Fakeradio} Any transfer between two nodes
\end{itemize}
\todo{Add why these were chosen?}
This list covers all major aspects of a topology test; transfer of bundles, setup and layout information, sending and receiving general packets (including the tree-sync packets), and some internal logic and processing when packets are sent and received.
With only these important events, it should be possible to locate and isolate issues related to transferring bundles and packets. 
From there, these filtered events should allow a tester to more easily utilise the vastly more expansive and detailed log file. 


Once the desired events had been determined, these then needed to be filtered within the program.
To achieve this, each line in the file is analysed to determine if it is important by examining what substrings the line contains.
A line is accepted if it matches specified criteria; for instance, a line within the Fakeradio process that contains the substring "neighbour has a bundle" would be considered important.

These lines are then sent to the relevant function to be formatted appropriately.

\subsection{Output format}
\begin{itemize}
    \item \#Original log file format
    \item \#How we get consistent format
    \item \#What the output format is
    \item \#Setup section up top
    \item Getting consistent timestamp (talk about issue with LBARD (. instead of :) T+, etc.)
    \item \#Show a screenshot
\end{itemize}

\subsubsection{Formatting lines}
The log files produced by the test framework follow a consistent structure.
The structure always follows the structure of:
\begin{itemize}
    \item the test details, 
    \item the output of the test framework, 
    \item output of Fakeradio, 
    \item output of \emph{each} LBARD process, and finally,
    \item the output of \emph{each} Servald process.
\end{itemize} 

This structure means that filtering each line becomes far simpler, since we only need to track which process (Fakeradio, LBARD, Servald) we are in, and then filter lines within that process that are relevant.
However the test framework log files have one major issue: the format of lines is not consistent between processes.
For instance, the typical Servald line may look like
\begin{center}
    \begin{lstlisting}[breaklines]
DEBUG:[511710] 18:15:34.372 overlay_mdp.c:859:_overlay_send_frame()  {mdprequests} Send frame 68 bytes    
    \end{lstlisting}
\end{center}
while a similar line in LBARD may look like 

\begin{center}
    \begin{lstlisting}[breaklines]
T+25138ms : Sending length of bundle 6A1A3379501553D1* (bundle #0, version 1596098704035, cached_version 1596098704035)
    \end{lstlisting}
        \todo{Make this font smaller?}
\end{center}


As can be clearly seen, there is a huge difference in format between these two lines, and as such these need to be formatted differently.
To achieve this, when a line is filtered as outlined in the previous section, the program returns an integer value representing the type of line that is to be formatted.
With this information, the appropriate function can be called for the line type, so that it can be formatted correctly.

For Servald lines this becomes trivial, simply use the \verb|sscanf| function to extract the important information from the line, format and write this to a variable using the \verb|sprintf| function, and then write this line to the output file. 
However, in the case of LBARD and Fakeradio lines, multiple issues arise due to the formatting of the log files.
The first of these issues is that several lines in LBARD are not timestamped with the time that they occurred, rather they are timestamped with the number of milliseconds since the program started.
This is an issue since this means that the log files can not be easily sorted by timestamp, and also will not be formatted with a consistent format with the other lines.
To fix this, the time that an LBARD instance is started is logged as this is in a normal timestamp format, and when an event with a 'T+' timestamp occurs, the number of milliseconds is added to the original timestamp to produce a timestamp. 
\todo{Fix these sentences}

The other issue with this method is that Fakeradio log lines often are spread over multiple lines.
This means that simply scanning and processing a single line will not produce all the necessary information.
However, when 
\todo{Finish this}

\todo{Add sorting the output}
\subsubsection{Output log file}

Once the log file is produced it follows a consistent format.
The simple log files format begins with the setup section.
The setup section lists all of the essential information for drawing a diagram of the network topology. 
It lists the test details, SIDs of each of the nodes, and all of the WiFi connections and Fakeradio rules. 
An example of the setup section can be seen in \figurename{ \ref{fig:chapter5SimpleLogSetup}}.

\begin{figure}
    \begin{centering}
        \includegraphics[width=15cm,height=20cm,keepaspectratio]{Figures/Chapter5-SimpleLogSetup.png}
        \caption{Setup section of the simple log}
        \label{fig:chapter5SimpleLogSetup}
    \end{centering}
\end{figure}
After the setup section each line in the log file is ordered by chronological order. 
The lines follow a simple and consistent structure.
\begin{center}
    \begin{lstlisting}[breaklines]
[Timestamp] [Process]:[Instance Letter] [Description]
    \end{lstlisting}
\end{center}
\todo{Maybe have these figures in a lstlisting themselves. Might make the text easier to read?}
This structure can be seen in \figurename{ \ref{fig:chapter5SimpleLogFormat}}
\begin{figure}
    \begin{centering}
        \includegraphics[width=15cm,height=20cm,keepaspectratio]{Figures/Chapter5-SimpleLogFormat.png}
        \caption{Format of simple log events}
        \label{fig:chapter5SimpleLogFormat}
    \end{centering}
\end{figure}


\subsection{Combining with test framework}
\begin{itemize}
    \item Automatically run - in finally section
    \item Need to specify command line arguments - input, output, etc.
    \item Output to log directory
\end{itemize}

\section{Generating a network diagram}
As an improvement to the outputs of the test framework, rendering a network diagram is highly useful, as it allows for testers to better understand the topology that they are testing.
In the test definitions, it is often difficult to understand what the network topology of a given test is.
This is due to the fact that the topology definitions - while easy to understand for a computer - are not particularly friendly for humans to understand, due to their relatively complicated definition.
The topology definition are split into two section, WiFi and Fakeradio, and lists connections only between two nodes - not the entire network.
\todo{Add more / fix this}
To render these diagrams, the layout will need to be extracted from the test definition, then a diagram drawn of this topology.
From this layout, the image will then need to be created and rendered.
There are two main solutions for creating the image: do it using a graphics library, or creating a DOT file and rendering it with Graphviz.
Using a graphics library would provide a high level of control over the creation of the image as graphical elements would need to be defined explicitly and as such are unlikely to have unforseen side-efects as could be expected with using a higher-level solution such as Graphviz.
That said, this would be considerably more complicated and harder to maintain than simply writing DOT files, then using the high-level tools provided by Graphviz to render this DOT file.
DOT files are text files that follow a specified syntax, and are used by Graphviz to render and create graphs.
These files are simple, and both human and machine readable. 
After weighing these options it was decided that despite the far higher level of control possible by using a low-level graphics library, the simplicity and maintainability of creating and rendering DOT files far outweighs this advantage.

\begin{itemize}
    \item Get layout
    \item Decide on DOT file format
    \item Get all combinations of links
    \item Add to dot file
    \item Render 
    \item Why we're hiding fakeradio/WiFI
    \item   (cos we want to animate, doesn't work otherwise)
    \item Show an output diagram (oooh fancy! pictures!)
\end{itemize}






\section{Animating the network diagram}

\subsection{Getting Major and Minor events}
\emph{This is just my thoughts as I was working through the issue. This is not going to be in the final thesis}
I am currently attempting to collect up all of the major (file transfer, show on diagram) and minor (important enough to be on diagram but not on topology) events.
Initially it was planned to get this out of the log file as the log file is processed and the simple log created. 
However, during implementation it was realised that as the original log file is not chronologically ordered, minor events will not easily be tied to a major event.
This is because the majority of the major events occur in fakeradio - when a packet moves from one node to the other.
However, fakeradio - in the log file - comes before LBARD or even Servald.
Thus, the major events that occur in fakeradio will not be tied to any minor event in LBARD or Servald.
There are two possible routes that can be taken to solve this. 
The first, is to simply process the simple log file AFTER it is created, and essentially copy/paste large amounts of code with minimal difference to make it work with processing major/minor lines.
The other alternative is to save all major and minor events while processing the simple log.
The major events can be stored in a struct as planned.
Minor events can simply be a specially formatted string - essentially the same as those in the simple log.

After the simple log is created, the program then sorts both the array of major events, AND the array of minor events.
A blank major event must be created before processing to allow for any minor events that occur BEFORE a major event 
With these both chronologically sorted, the program simply adds any minor events that occur BEFORE a major event to the major event, then once one appears that is LATER than the major event, switches to the next major event.
This loops through until each major and minor event has been processed.
If any minor events are left after the last major event it is added to a blank major event (so nothing gets left off).
A maximum of \emph{n} minor events can be assigned to a major event before it switches over to a new blank major event.

A new array of major events is created after sort. This doubles up on memory, but means that blank events can be added without overwriting other major events.

Does not include broadcast messages because they don't contain enough information.

\subsection{Rendering all of the images}
\begin{itemize}
    \item How we render one
    \item Do in a loop for each of them
    \item LaTeX template, etc.
\end{itemize}

\section{Summary}
\textbf{Add link to next section} 
% Chapter Template

\chapter{Visualising tests in a real network} % Main chapter title

\label{Chapter6} % Change X to a consecutive number; for referencing this chapter elsewhere, use \ref{ChapterX}

%----------------------------------------------------------------------------------------
%	SECTION 1
%----------------------------------------------------------------------------------------

\section{Main Section 1}
Chapter 6
- Apply but in test network
- Real world tests

%-----------------------------------
%	SUBSECTION 1
%-----------------------------------
\subsection{Subsection 1}


%-----------------------------------
%	SUBSECTION 2
%-----------------------------------

\subsection{Subsection 2}


%----------------------------------------------------------------------------------------
%	SECTION 2
%----------------------------------------------------------------------------------------

\section{Main Section 2}

% Chapter Template

\chapter{Discussion of results} % Main chapter title

\label{Chapter7} % Change X to a consecutive number; for referencing this chapter elsewhere, use \ref{ChapterX}

In this chapter the fourth research question "\fourthRQ" will finish being answered with an evaluation of the tools and improvements developed throughout this project.
Each of these tools and improvements will be summarised and the strengths and limitations critically analysed.
Further, the bugs and issues in \texttt{LBARD} that have been uncovered or demonstrated and reliably reproduced will be discussed.

\section{Evaluation of framework expansion}
Three main improvements were made to the \texttt{LBARD} test framework, with each significantly expanding or improving the capabilities of the test framework.

The first of these improvements was the implementation of per-node configurations.
This improvement allows for an increased amount of control around how emulated networks behave.
This allows for nodes in a test to have different interfaces defined allowing for more flexible network topologies than the original WiFi- or radio- only network topologies.
With the development of this, the goals of the Serval project — to create reliable and cost-effective communication — have been furthered, with the ability to now comprehensively test a far greater range of Serval networks than was previously possible.

The next improvement that was made to the test framework is the development of new \texttt{setup} functions to allow for easier testing of networks of arbitrary size.
This improvement is more of a quality of life improvement than a significant feature increase, however its impact to the testing process of the Serval team is non-negligible.
Before this improvement had been made, tests were generally limited to using either 4 or 8 nodes in a test. 
This is not due to any limitation in the test framework itself, as the framework is capable of handling up to 26 nodes (A-Z), but simply due to testers using the \texttt{setup} functions available to them which only setup and started 4 or 8 Serval nodes.
However, with the development of the \texttt{setupN} function, an arbitrary number of Serval nodes can be deployed for the test, allowing for more flexibility in test design, without requiring testers to write a function to deploy the exact number of nodes that they require.

Finally, 8 different tests were written using these newly developed improvements.
These tests focused on testing several topology-specific Serval network layouts.
These tests however do have some limitations. 
First, there is minimal variety in the layout of the newly developed tests, with half of the tests all featuring some variation on a simple chain of nodes.
Further, these tests predominately tests networks using one file at the start node and concluding when this file reaches a specified destination node.
While there is some variation in the specific layout, these tests present only a mild addition to the Serval test coverage.
However, it is hoped that these added tests serve as a useful starting point for future test developers to further test various Serval topologies.


\section{Evaluation of developed tools}

\subsection{Creation of simple log}
The first tool that was developed for the Serval project is that of the simple log generation.
This tool allows for the log file produced by a test to be simplified, formatted, and sorted into chronological order.

With this tool, the Serval team are able to more effectively determine where an issue occurs during a test without needing to search through the unordered, non-formatted original log file.
With this tool the Serval team are able to look through the simplified log file chronologically, and develop an understanding of how the network was behaving at any given time.

The generation of the simplified log file does have some limitations however.
First, it may not produce enough output information to the tester to diagnose an encountered issue.
Fortunately this is not necessarily what the use of this simple log file is; this simplified log file is aimed at locating where issues are occurring in the test, at which point the tester can then use this information to more effectively use the initial log file.
This generation of the simple log file is further limited by the amount of information that it filters.
As it stands, the simple log file only keeps specific items from the initial log file.
As such, if the capabilities of \texttt{LBARD}, \texttt{servald}, or \texttt{fakeradio} are extended, this would need to added to the functionality of the program.
This should not present a major issue in the future, as implementing this should just prove to be a matter of expanding what lines the program does not filter out.

\subsection{Network traffic visualisation}
The network visualisation tool was developed for analysing Serval networks  after tests.
This tool had two main outputs for visualising the network traffic: ASCII and LaTeX/PDF outputs.
Both of these visualisation methods share some key features.
These features are as follows:
\begin{itemize}
    \item \textbf{Does not modify log lines} from the lines added to the simple log. 
    This allows for testers to use the generated diagrams as an aid to the simple log by visually determine where issues may be occurring, then investigating the log file for more details. 
    \item \textbf{Shows log lines associated with a major event}. 
    \item \textbf{Display a bundle bitmap} allowing for testers to determine what pieces of a bundle have not been sent or have been sent several times to a node.
    \item \textbf{Display test information}.
    \item \textbf{Display test statistics}. 
    Gives information about number of major/minor events, bundles, and malformed lines that are encountered in the original log file.
    \item \textbf{Shows overview of test events} throughout the test, helping testers to determine what is occurring at a given point of time.
\end{itemize}
With these features, Serval Testers are now able to visualise the network traffic during a test, and gain a greater insight into the functionality of \texttt{servald} and \texttt{LBARD} during their tests.

\subsubsection{ASCII}
The ASCII output format has some strengths and weaknesses that are specific to that format in addition to the features listed above.
The ASCII output is able to function without any additional external dependencies as it is using standard C libraries to display this output.
This allows for high levels of portability of this application, allowing for all Serval testers to use this program without requiring testers to install applications such as GraphViz or LaTeX tools.
In the ASCII output, testers are able to filter out the display of minor events and only display the major events.
This is useful for testers as they are able to quickly see the transfers that occur between nodes.

However, the ASCII output is limited as it does not ever show an entire ASCII representation of the entire network topology.
This may hinder the ability of testers to visualise the network topology as each major event.
An ASCII network representation could be implemented in the future to mitigate this limitation.
This is covered in further details in Section \ref{section:futureWork}/

\subsubsection{Diagrams}
The PDF generation tool allows for Serval testers to easily visually determine the network layout of a tested Serval layout, and easily understand how data is transferred throughout the network.

The LaTeX PDF generation tool has several strengths that allow it to serve as a useful tool for the Serval team.
The first of these is the visual nature of the output; testers are shown the entire network topology in an easily understandable diagram with the network traffic of each major event shown on the diagram.
This allows for testers to easily understand where traffic is being sent throughout the network, and isolate where issues are occurring when this data is not being sent.
This is also helpful for demonstrating core Serval and \texttt{LBARD} concepts to new contributors to the Serval Project.
As the Serval Project is an open-source project, this is crucial for expanding the Serval team and ensuring the longevity of the project.

The generated PDFs are also very portable as it is a simple PDF file, allowing for Serval team members to easily share the PDF with other team members and easily indicate where the issues are occurring.

The PDF generation does have some weaknesses.
The first of these is that the diagram generation is not perfect and may produce some non-optimal layouts.
While the diagrams generated are correct to the test definition, these diagrams produced may have lines that overlap and are not organised in a particularly logical layout.
This can be improved through further work in the diagram generation.
This is detailed in Section \ref{section:futureWork}.

Further, when several major events occur at the same time these are not all shown on the diagram.
This is due to the assumption that each major event occurs at a single unique point of time.
In normal usage, this does not often occur with unrelated major events.
For some events, such as \texttt{LBARD} sending a piece of a bundle at the same time as it sends a synchronisation message, these will be displayed as different major events.
It is not expected that this will cause issues with debugging tests, however if this functionality is later required this should require minimal extra work.

\section{Debugging Serval}
Throughout the process of this thesis several bugs and memory issues have been uncovered and reliably reproduced.
These issues have occurred in both the emulation and real-world tests.

The first of these issues is the reliable reproduction of a bug where a \texttt{LBARD} node will continue resending bundle pieces to a neighbour despite the neighbour having received an entire copy of this bundle and reporting internally that this bundle has been received.
This issue has been encountered in both the RFD900 and CodanHF radio interfaces, as well as in real-world field tests using real CodanHF radios that have been analysed with these tools.
Reliable reproduction of this bug is essential for determining the cause and verifying the fix of this bug.
This bug was reported to the first supervisor of this thesis who believed they had fixed this issue, however when the appropriate test was re-run with the updated software, the bug still remained.
Using the tools developed throughout this thesis allowed the Serval team to determine that this bug must be occurring in the tree synchronisation protocol of \texttt{LBARD} as the tools indicated that this occurred when different radio modules were used.

A further issue that has been uncovered through this thesis is the delayed reporting of bundles being received.
In tests, it appears that there is an approximately 3 second gap between when \texttt{LBARD} reports that all the pieces of a bundle have been received and when the bundle is marked as added to the database.

Additionally, several memory issues in \texttt{LBARD} and Serval-DNA have been located through the use of the simple log and diagram generation tool.
As this tool parses the initial log file, it attempts to match lines to specific filters.
However sometimes a line matches an initial filter but fails to be processed properly by the program as the format does not match the proper format that lines that match this filter follow.
This is due to memory leaks in \texttt{LBARD} and Serval-DNA where enough of the line matches the filter, but due to a memory issue, other parts of the line are interrupted by other log lines that have leaked into this space in memory.
To handle this, the tool attempts to format these lines, but if this formatting fails reports this as a malformed line.
In one such test over 1,100 lines in a field test were reported as malformed.


\section{Summary}
This chapter provided a detailed evaluation of the improvements made to the test framework and the tools developed, and served as an answer to the final research question "\fourthRQ".
The improvements made to the framework were evaluated, and their strengths and limitations detailed.
These improvements were additionally analysed on their use to the Serval Project as a whole.
This chapter further analysed the tools throughout this project, and detailed their strengths and limitations.
Finally, the bugs and issues that have been uncovered in \texttt{LBARD} and the test framework were described, and the process for uncovering them listed.

In the next chapter, this thesis will be summarised, with each of the research questions reiterated and a summary of the answers provided
Further, the future work is investigated, and the improvements that could be made to the tools developed in this thesis outlined.
% Chapter Template

\chapter{Conclusion} % Main chapter title

\label{Chapter8} % Change X to a consecutive number; for referencing this chapter elsewhere, use \ref{ChapterX}

%----------------------------------------------------------------------------------------
%	SECTION 1
%----------------------------------------------------------------------------------------

\section{Main Section 1}
Chapter 8
- Conslusion


%-----------------------------------
%	SUBSECTION 1
%-----------------------------------
\subsection{Subsection 1}


%-----------------------------------
%	SUBSECTION 2
%-----------------------------------

\subsection{Subsection 2}

%----------------------------------------------------------------------------------------
%	SECTION 2
%----------------------------------------------------------------------------------------

\section{Main Section 2}



%----------------------------------------------------------------------------------------
%	THESIS CONTENT - APPENDICES
%----------------------------------------------------------------------------------------

\appendix % Cue to tell LaTeX that the following "chapters" are Appendices

% Include the appendices of the thesis as separate files from the Appendices folder
% Uncomment the lines as you write the Appendices

% \include{Appendices/AppendixA}
%\include{Appendices/AppendixB}
%\include{Appendices/AppendixC}

%----------------------------------------------------------------------------------------
%	BIBLIOGRAPHY
%----------------------------------------------------------------------------------------

\printbibliography[
	heading=bibintoc,
	title={References}
	]

\end{document}  
