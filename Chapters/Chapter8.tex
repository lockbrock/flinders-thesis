% Chapter Template

\chapter{Conclusion and Future Work} % Main chapter title

\label{Chapter8}

\section{Conclusion}
Throughout this thesis four questions have been addressed.
The first question, "\firstRQ" was answered in Chapters 2 and 3.
In Chapter 2, an in-depth review of the existing literature was conducted, and an overview provided of Mesh Networks, the Serval Mesh Network and associated technologies, an overview of Network Testing, and a review of popular features in network emulators and simulators.
Chapter 2 also answered the second research question "\secondRQ".

From there, in Chapter 3 an in-depth investigation into LBARD and the test framework was conducted.
In this chapter, the Serval-DNA and LBARD test frameworks were investigated and the processes behind defining and running tests was outlined. 
With the analysis of the test framework conducted, the limitations of the framework were discussed.
From these limitations, the suggested improvements to be added to the test framework were outlined, and the specific requirements for these improvements were discussed.
This answered the third research question "\thirdRQ".
The improvements that were suggested to be added to the test framework were:
\begin{itemize}
    \item Add the capability to communicate over both radio and WiFi in a test
    \item Add more topology-focused tests
    \item Create a simpler, consistently formatted log file for easier test analysis
    \item Implement a graphical output showing the network traffic during a test
    \item Validate the accuracy of the test framework by comparing it against real-world tests
\end{itemize}
With the exception of the final improvement, each of these was implemented throughout this thesis.

In Chapter 4 the fourth research question "\fourthRQ" began to be answered with the extension of the test framework.
The first improvement to be made to the test framework was the implementation of per-node configuration, allowing for tests that feature both WiFi and radio links between nodes.
An additional improvement was the implementation of a function that allows Serval testers to set up and start the exact number of Serval nodes that a test required.
With these improvements, 8 new topology-focused tests were added to the framework, with the goal that this would provide a basis for Serval testers to develop more topology tests.

In Chapter 5 the fourth research question continued to be answered with the development of a tool to generate simpler log files and generate graphical representations of network traffic through a test.
The simple log file produces chronologically ordered and consistently formatted log files of crucial log lines from the original log file produced by the test framework.
Two graphical representations were developed, an ASCII output, and a PDF output with diagrams showing network traffic throughout the entire network.
Each of these outputs shows the transfer of data between Serval nodes and details any minor events that are occurring that are related to this data transfer.

Chapter 6 detailed how these tools were adapted to model and analyse real-world Serval test results.
With this adaption made, the calculation of bundle bitmaps could be developed, allowing for the display of a count of the number of times a piece of a bundle had been sent to a node, allowing for a detailed analysis of bundle transfer throughout the network by testers.
In the original plan of this thesis, this chapter then performed a comparison between an emulated version of the Serval Test Network at Flinders University, Tonsley Campus, however due to the COVID-19 pandemic this was unable to be completed.
The methodology to perform this comparison and the criteria that should be analysed was outlined for the use of further development and analysis by the Serval team.

Finally, in Chapter 7 each of the improvements made to the test framework and the additional tools developed were analysed in detail.
Each of these tools and improvements were evaluated on their strengths and weaknesses, and their contributions to the Serval Project discussed.
In Chapter 7, these tools were further evaluated by discussing the bugs that were uncovered or demonstrated using these tools throughout the process of this project.
Chapter 7 finished answering the fourth research question "\fourthRQ".



\section{Future Work}
\label{section:futureWork}
Overall, this project has met the goals of this thesis.
However, some minor improvements and additional features could be added to the tools in the future.
The first of these is the auto generation of network layouts.
This should require a mild effort to implement, and would remove the necessity to manually specify the network layout for real-world tests.
To implement this, whenever a node registers a neighbour over LBARD or WiFi, then this would need to be added as a neighbour of that node.
With this, a connected graph could be generated and used to determine the network layout of the test.

An additional improvement that could be added is improving the diagram generation for the PDF.
As outlined in Chapter \ref{Chapter5}, the GraphViz \texttt{neato} tool is currently used for generating the diagrams as it allows for nodes to be pinned in place. 
This is essential for ensuring that diagrams are in a consistent layout throughout the PDF. 
However, for some network layouts neato does not place nodes in a logical layout and some 

This can be fixed by initially rendering the diagrams using the GraphViz \texttt{dot} tool which places nodes in a hierarchical structure, creating more logical diagrams but does not allow for nodes to be pinned in place.
Then, the \emph{x} and \emph{y} coordinates of each node can be extracted from this DOT diagram and then these coordinates set for each node in the \texttt{neato} diagram, allowing for these nodes to be manually placed and then pinned in place in a logical layout.
This was not done in this thesis due to time limitations.

\section{Summary}
This thesis project has improved the testing framework of the Serval Mesh Network and developed tools to assist with the debugging and testing of the Serval Mesh.
These tools have already proven useful to the team by uncovering and demonstrating several issues in the Serval Mesh.
With these tools, the Serval Project team are able to further their goals to create an infrastructure-less mesh network for use in disaster struck areas to save lives.