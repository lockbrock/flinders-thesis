% Chapter Template

\chapter{Description and analysis of existing LBARD framework} % Main chapter title

\label{Chapter3} % Change X to a consecutive number; for referencing this chapter elsewhere, use \ref{ChapterX}

%----------------------------------------------------------------------------------------
%	SECTION 1
%----------------------------------------------------------------------------------------

\section{Description of existing LBARD framework}
Chapter 3
Description and analysis of existing LBARD framework
\begin{itemize}
    \item Go into much more detail than lit review
    \item Since we are extending, need to fully explain and understand
    \item Analysis of log files
    \item how it functions
    
\end{itemize}


\subsection{Aims of test framework}
What it /wants/ to be doing

Why it exists





\subsection{Functionality}
What it currently does

The features it has

Technical infomation
\begin{itemize}
    \item What inputs it takes
    \item How the tests work
    \begin{itemize}
        \item Setting up virtual devices
        \item Simulates links and traffic between
        \item Logs data transfer along those links
    \end{itemize}
    \item 
\end{itemize}


\subsection{Outputs}
Log files

Two types: network transfer, and setup/hardware detection


Format of them



\section{Limitations}
What it doesnt do
\begin{itemize}
    \item No ability to have dual link types; ie. a node cant send/receive WIFI as well as send/receive UHF. Means that all topologies are limited to a single type of radio type. \emph{Not realistic at all for a real-world situation}
    \item Cannot currently use the serval-dna wifi type. All are currently using the rhizome method
\end{itemize}

\section{How to improve}
Talk about what I'll be doing to improve the framework

What features it needs to suit this honours


\begin{itemize}
    \item Add new lbard/tests file specifically for topologies: no point in rendering .dot files for the small setup stuff
    \item Add to topology test file the ability to have multiple link types: we're doing combination radio types in the real world so we need to cover that
\end{itemize}
