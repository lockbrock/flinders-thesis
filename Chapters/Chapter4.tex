% Chapter Template

\chapter{Extending the Test Framework} % Main chapter title
\label{Chapter4}
In this chapter, two of the improvements that were suggested in the previous chapter will be implemented, and the fourth research question "\fourthRQ" will begin to be answered.
The first improvement that will be implemented is adding the functionality to have tests with WiFi running alongside radio.
The next improvement to be implemented is the addition of several new tests that will use this new functionality.

\section{Joint WiFi \& Radio Tests}
As discussed in the previous chapter, the test framework lacks any functionality to test networks with WiFi alongside radio links. 
Without this functionality, Serval testers are limited to a relatively small sample size of network topologies, limiting their abilities to test a vast range of Serval functionality.

To implement this functionality, per-node configuration needs to be added to the test framework, allowing testers to define what interfaces each node will be using.
Then, the ability to define network topologies needs to be implemented to allow for these networks to be created and tested.
This already exists in the case of the \texttt{fakeradio} networks as discussed in the previous chapters, however for WiFi nodes it appears to only be possible to have nodes communicating with every other node without any defined topology forming a full-connected Mesh network.

To ensure that these changes to the test framework do not affect the existing \texttt{LBARD} framework tests, two new files will be created.
The first file (\texttt{topologies}) is the new bash file where the new tests will be defined.
Next, as these topology tests will need to use  and overwrite some test definitions from the original \texttt{LBARD} framework \texttt{testdefs.sh} file, a new test definition file (\texttt{topology\_testdefs.sh}) will be created that extends these functions.

In this section two approaches will be attempted to define the WiFi topology and configurations.
The first of these is to not enable WiFi on a node unless specifically defined, as is attempted in Section \ref{subsection:DefiningInterfaces}.
While this approach was achievable, after discussion with the first supervisor of this thesis, the recommendation was made to use functionality from the Serval-DNA test framework that allowed for WiFi network topologies to be defined by connecting their WiFi interfaces.
With this approach, WiFi could be enabled on every node in the network, however these nodes would not be connected over WiFi unless specifically defined in the test definition.
This is the approach that was taken in Section \ref{subsection:DefiningWiFi} and was used for the remainder of this thesis.
For the sake of completeness, both the initial and the final methodology will be described.


\subsection{Defining interfaces per node}
\label{subsection:DefiningInterfaces}
To allow for nodes to support different network interfaces, the ability to specify node-specific configurations needed to be implemented.
Prior to this implementation, each Serval node in a test used the same configuration as defined in the boilerplate \texttt{setup} function provided by the framework.

The first step to completing this is to determine what configuration is required for a given node.
Two possibilities exist for this; manually define the configurations for each node in the test definition or automatically determine which configuration is required for each node.
Since each node that will be connected by \texttt{fakeradio} needs to be defined already, it was decided to use this information to determine what configuration the node needed.

Each test that uses \texttt{fakeradio} contains a string in the setup function defining the links between nodes.
The string defines the links in the format 

\texttt{allow between [n],[m];} 

Where n and m are the 0-indexed serval-dna instance number.

Every node that will exist in a \texttt{fakeradio} topology is required to be listed in this string.
As such, this string lists every node that requires a \texttt{LBARD} interface to be defined.
As the WiFi functionality does not need to have the topology defined in such a way it is necessary to define a new string, listing the nodes that require WiFi to be enabled. 

In the test definition, both of these strings are passed to the \texttt{start\_instances} function.
To determine what nodes to enable the configurations on, the node information needs to be stripped from the string.
To achieve this, both of these strings are altered to extract the 0-indexed instance numbers which are then assigned to an array of integers.
This results in two arrays (WiFi and \texttt{LBARD}) of the instance numbers.
With this information, the configurations that are required for each node can be programmatically determined.

To achieve this, a new array is created that contains the combination of these two arrays with valid configurations being "\texttt{wifi}", "\texttt{lbard}", or "\texttt{wifi lbard}".

When each node is started, this array is checked and the appropriate configuration is applied to that node.
For nodes that require \texttt{LBARD} to be started, the configuration required for Serval-DNA is enabling the HTTP API for \texttt{LBARD} to connect to, and setting the username and password for this API.
For nodes that require WiFi to be enabled, the configuration required can be seen in \figurename{ \ref{fig:chapter4WifiConfig}}.

\begin{figure}
    \begin{centering}
\begin{lstlisting}[basicstyle=\small, breaklines, frame=single]
set server.interface_path [SERVALD_VAR]
set interfaces.[INTERFACE].socket_type [SOCKET_TYPE]
set interfaces.[INTERFACE].file dummy[INTERFACE]/[INSTANCE_NAME]
set interfaces.[INTERFACE].type [TYPE]
set interfaces.[INTERFACE].idle_tick_ms 500
\end{lstlisting}
        \caption{WiFi configuration}
        \label{fig:chapter4WifiConfig}
    \end{centering}
\end{figure}

With this implemented, WiFi and \texttt{LBARD} can be enabled on a per-node basis, allowing for nodes to connect over WiFi or \texttt{LBARD}.
However, this method is limited in that it does not allow for any configuration of a WiFi topology as every node will use the same WiFi interface, and so every WiFi node will form a fully-connected mesh by default.
It was at this stage that the first supervisor advised to change the methodology of defining and configuring WiFi nodes.
This is outlined in Section \ref{subsection:DefiningWiFi}.


\subsection{Defining WiFi topologies}
\label{subsection:DefiningWiFi}
As mentioned previously, the methodology for enabling WiFi configurations was changed due to an unforeseen complication in the planned method outlined in the previous section.
To mitigate this, the method outlined in the previous section was partially abandoned and an approach from the Serval-DNA framework was used.

In the Serval-DNA test framework, each WiFi interface is essentially a sub-network, with nodes within a specified interface only able to communicate with nodes within the same interface as themselves.
Thus, to define a complicated network topology it is necessary to restrict which interfaces WiFi nodes can communicate on.
In the Serval-DNA test framework, non- fully-connected WiFi networks are created by manually defining the WiFi interfaces that a specified \texttt{servald} interface is connected to.


To implement this, several changes needed to be made.
The first is alter the functions added in the previous section to only handle \texttt{LBARD} information.
This is since WiFi will now be enabled for every node, however each will only be connected to a single interface by default.
This interface will be unique for every node, with the interface id given by the index of the node plus 100.
It was believed that this should be high enough for normal usage of the test framework.

With this implemented, nodes can still have \texttt{LBARD} enabled if defined in the test definition as previously outlined, but will now always have a WiFi interface enabled but not connected to any other WiFi enabled nodes.
To connect nodes via WiFi they must now be connected to a specific WiFi interface with the command
\begin{lstlisting}[numbers=none, basicstyle=\small]
foreach_instance +[Node1] +[Node2] add_servald_interface [interface]
\end{lstlisting}
Where \emph{Node1} and \emph{Node2} are single node characters, and where \emph{interface} is a positive integer.
This can be seen in \figurename{ \ref{fig:definingInterfaces}}.

\lstset{language=bash,
showstringspaces=false,
numbers=left,
}

\begin{figure}[h!]
    \begin{centering}
        \begin{lstlisting}[breaklines, frame=single]
setup_DualType(){
    # A -r-> B -w-> C -r-> D
    # No radio connection between 1(B) and 2(C)
    lbardparams="allow between 0,1; allow between 2,3; deny all;"

    setup 0 0 0 "\${lbardparams}"
    # Connect 1(B) and 2(C) via a wifi/rhizome interface of number 1
    foreach_instance +B +C add_servald_interface 1

    # Insert a file to server A
    set_instance +A
    rhizome_add_file file1 50
}
        \end{lstlisting}
        \caption{Example of a test setup function}
        \label{fig:definingInterfaces}
    \end{centering}
\end{figure}

Using this method, a node is able to communicate with every other node that uses the same WiFi interfaces as it.
As a node can be connected to multiple WiFi interfaces, this allows for network topologies that are more complicated than simple fully-connected mesh networks, such as chains of WiFi nodes by enabling interfaces in such a way that no node is communicating with more than two neighbours.


\section{Improving Topology Definition}
To assist with creating complex network topologies, a function \texttt{setupN} was implemented that allowed for an arbitrary number of \texttt{servald} and \texttt{LBARD} instances to be created and networked together.
This function was required as the \texttt{LBARD} test framework lacked any way of defining an arbitrary number of nodes and instead only provides function to start 4 or 10 \texttt{servald} devices.
A function that can create and start an arbitrary number of nodes allows for greater flexibility in defining network topologies.

This \texttt{setupN} function accepts four parameters:
\begin{itemize}
    \item a string of node characters, 
    \item a string of radio types ('none' if \texttt{LBARD} is not to be started on the relevant node), 
    \item the \texttt{fakeradio} parameters (i.e. packet loss), 
    \item and finally the parameters for each \texttt{LBARD} instance.
\end{itemize}

This function then starts a \texttt{servald} process on each of these nodes, and as described in previous sections, starts \texttt{LBARD} as required.
Once \texttt{servald} and \texttt{LBARD} is running on every relevant node with the appropriate configurations as outlined in the previous sections, the function then starts \texttt{fakeradio} as necessary.

By using this function, testers are able to define and test the exact number of Serval nodes that they require in their test, and with the capability to define exact parameters of the test — such as defining which radio to use on a per-node basis and packet loss — this relatively simply function provides a greater range of flexibility.
Furthermore, this function when combined with the implementation of mixing WiFi and \texttt{fakeradio} interfaces allows for Serval testers to test a far greater number of Serval networks with minimal additional code required.


\section{Adding more topologies tests}
With the improvements made to the \texttt{LBARD} test framework, several new topology tests can be implemented to test the functionality of \texttt{LBARD} in various different network topologies.
The implemented topology tests do not all strictly need to feature WiFi alongside radio, as any test which focuses on a specific network topology that did not exist previously represents an expansion of the test framework.
All of these tests were implemented using the functionality developed throughout this chapter, and followed the process of defining tests as outlined in \chaptername{ \ref{Chapter3}}.

The tests that have been implemented are found in the following table.
\begin{table}[h!]
    \centering
    \begin{tabularx}{\textwidth}{l|X}
        \textbf{Name:}  &  \textbf{Description} \\ \hline
        DualType            &   A chain of 4 nodes, with A and B, and C and D connected with a RFD900 interface, and B and C connected over WiFi. A has a bundle. Finishes when D receives bundle \\ \hline
        RFD900 Chain    &   A chain of 10 RFD900 connected nodes. A has a 100 byte bundle. Finishes when J receives the bundle   \\ \hline
        WiFi\_RFD900\_Chain\_Short & A chain of 5 nodes with alternating UHF/WiFi links. A has a 100 byte bundle. Finishes when F receives the bundle. \\ \hline
        Wifi\_RFD900\_Chain & A chain of 5 nodes with alternating UHF/WiFi links. A has a 100 byte bundle. Finishes when J receives the bundle.     \\ \hline
        LBARD\_Big\_File\_Chain & 4 RFD900 nodes in a chain. A has a 5kb file. Ends when D receives the file.  \\ \hline
        Lab\_File   & Emulated Tonsley lab network. Node A (Lab node) has a file. Ends when H (node A3) receives the file.  \\ \hline
        Lab\_Message& Tonsley lab network topology. Simulates a MeshMS conversation between C (Tonsley node 9) and H (Tonsley node A3). Ends when C receives the final message from H.   \\ \hline
        SplitRFD900 & 5 RFD900 nodes. A can communicate with B, C, and D. E communicate with B, C, D only. A has a 50 byte file. Ends when E receives the whole file. \\ \hline        
    \end{tabularx}
\end{table}

\pagebreak


The additional tests represent a significant addition to the testing coverage of the test framework.
While the developed tests are not exhaustive, they provide groundwork for the further expansion of the test coverage.
The tests cover a wide range of test cases, with the developed tests focusing on how Serval networks handle communication through complicated network topologies.
This is most obvious in both of the Lab tests. These tests model the real-world lab network that the Serval team have set up at the Flinders University Tonsley campus.
These tests also provide the basis of comparison to real-world Serval networks as outlined in \chaptername{ \ref{Chapter6}}.
In these tests, the reliability of \texttt{LBARD} is tested, particularly in the \texttt{LBARD\_Big\_File\_Chain} test, where a large file is sent across several RFD900 hops.
Further, these tests analyse how \texttt{LBARD} handles a single node receiving bundle pieces from several nodes at a time, notably in SplitRFD900 and both of the Tonsley lab tests.


\section{Summary}
In this chapter the testing capability of the Serval Mesh Network has been significantly increased, and the fourth research question "\fourthRQ" has been answered in part.
This was done through three different methods.
The first was the implementation of allowing the test framework to handle multiple different interfaces.
With the implementation of this, future tests are able to support a mixed network topology, with the ability now present to have tests where nodes can communicate over both WiFi and radio. 

In this chapter, the test framework was further improved by increasing the ease of defining networks using the framework.
This was done by providing the ability to easily define the exact number and configuration of nodes for the test.
This allows testers a far greater level of control over the parameters of the test, thus allowing them to test networks more effectively.

Finally, these improvements to the test framework were utilised to add significantly more topology-focused tests.
Using these improvements to expand the test coverage serves a dual purpose: first, the Serval network is more thoroughly tested, and secondly, the newly created improvements are themselves tested by being used and their usefulness demonstrated.

This chapter has begun implementing the improvements that were recommended in Chapter \ref{Chapter3}.
In the next chapter, the next of these improvements, namely the implementation of simple log and diagram generation will be implemented and continue answering the fourth research question.
