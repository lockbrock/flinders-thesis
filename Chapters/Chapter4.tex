% Chapter Template

\chapter{Extending the Test Framework} % Main chapter title
\label{Chapter4}

%----------------------------------------------------------------------------------------
%	SECTION 1
%----------------------------------------------------------------------------------------

\begin{itemize}
    \item Modify test framework
    \item Modify fakeradio to output layout
    \item Get dual link types working
    \item Be able to define different WiFi layouts
    \item 
\end{itemize}


\begin{itemize}
    \item Currently, no way to easily determine what topology is in machine readable format
    \item What info is needed for this 
    \item How can we represent it?
    \item Why format?
    \item JSON vs CSV
    
\end{itemize}


%-----------------------------------
%   SECTION 1
%-----------------------------------
\section{Joint wifi \& radio tests}
A key part of the test framework that was missing was support for networks with both wifi and radio link types. 
This presents a major limitation  to the test framework as real Serval networks would feature both of these network links.
To implement this feature, several additions need to be made to the test framework.

First, the ability to define which link simulation will be used for each nodes needs to be defined.
This allows test definition writers to define the functionality(??) of each of the nodes. 
For instance, test writers are able to define if a specific node in a topology is using the fakeradio tool or simply acting as wifi, or even, both. 

Next, the ability to define specific network topologies will need to be added.
This already exists in the case of the fakeradio networks as discussed in the previous chapters, this is not easily done with the simulated wifi tool.

Once these two pieces of functionality have been added to the test framework, developers will be able to test considerably larger and more complicated network topologies with ease.
With this expanded test framework the Serval team will be able to significantly improve their knowledge of the Serval network; complicated, larger, and reproducable serve a huge benefit to the goals of the Serval team. \emph{rewrite, this is bad}



What needed to be done for this?
\begin{itemize}
    \item Add capability to define interface to use
    \item Strip definitions to get lists of what interfaces for each node
    \item For each node, add the interfaces that are in lists
    \item For each node, add appropriate configuration
    \item Pass lbard rules to fakeradio as per normal
\end{itemize}


\subsection{Subsection 2}


%----------------------------------------------------------------------------------------
%	SECTION 2
%----------------------------------------------------------------------------------------

\section{Retreiving network layouts}
To be used in next section when we are creating diagrams from the network layouts
